\documentclass[12pt, a4paper, oneside]{article}
\usepackage{seephy}
\title{\textbf{Computational Physics(A) \\Assignment 2}}
\author{Chon Hei Lo\thanks{Email: see.looooo@stu.pku.edu.cn; StudentID: 2000012508} (罗俊熙) \\ School of Physics, Peking University}
\date{\today}
\linespread{1.5}

\newcounter{problemname}
\begin{document}

\maketitle

\begin{center}
\textit{注1: 此作业的解答如无说明,统一使用爱因斯坦求和约定。}
\end{center}
\section{Problems \& Solutions}
% ==============================Problem 1==================================
\subsection{Chebyshev 近似(15pt)}
給出$f(x)$的3阶和5阶的 Chebyshev 近似展开,讨论及作图比较。
$$f(x)=x^6+3x^5+4x^4+\frac13x^3+2x^2+x-10,\quad (-1\le x\le 3)$$


% ==============================Solution 1==================================
\textbf{Solution:}為了將定義域$[-1,3]$映射到$[-1,1]$,令$y=\frac{x-1}{2}$。對於3階來說,有:
$$c^{(4)}_0=\frac{1}{4}\sum_{k=0}^{3}f\ab[2\cos\ab(\frac{\pi(k+1/2)}{4})+1]$$
$$c^{(4)}_1=\frac{2}{4}\sum_{k=0}^{3}\cos\ab(\frac{\pi (k+1/2)}{4})f\ab[2\cos\ab(\frac{\pi(k+1/2)}{4})+1]$$
$$c^{(4)}_2=\frac{2}{4}\sum_{k=0}^{3}\cos\ab(\frac{2\pi (k+1/2)}{4})f\ab[2\cos\ab(\frac{\pi(k+1/2)}{4})+1]$$
$$c^{(4)}_3=\frac{2}{4}\sum_{k=0}^{3}\cos\ab(\frac{3\pi (k+1/2)}{4})f\ab[2\cos\ab(\frac{\pi(k+1/2)}{4})+1]$$
使用程序\file{3-1.py}計算得到:
\begin{align*}
    c^{(4)}_0&=369.3\\
    c^{(4)}_1&=664.0\\
    c^{(4)}_2&=444.0\\
    c^{(4)}_3&= 204.7
\end{align*}
又有$T_0(y) = 1, T_1(y)= y, T_2(y) = 2y^2 - 1, T_3(y) = 4y^3 - 3y$
所以
\begin{align*}
    S(x) &= \sum_{m=0}^3c^{(4)}_mT_m(\frac{x-1}{2}) \\ 
         &= 369.3 + 664.0(\frac{x-1}{2}) + 444.0\ab[2\ab(\frac{x-1}{2})^2-1] + 204.7\ab[4\ab(\frac{x-1}{2})^3-3\ab(\frac{x-1}{2})] \\
         &= 102.35x^3 - 85.05x^2 - 112.0x + 20.0
\end{align*}
注,由於$c^{(4)_m}$都進行了四捨五入,所以最後的結果可能會有一點誤差。\\
對於5階來說,有:
$$c^{(6)}_0=\frac{1}{6}\sum_{k=0}^{5}f\ab[2\cos\ab(\frac{\pi(k+1/2)}{6})+1]$$
$$c^{(6)}_1=\frac{2}{6}\sum_{k=0}^{5}\cos\ab(\frac{\pi (k+1/2)}{6})f\ab[2\cos\ab(\frac{\pi(k+1/2)}{6})+1]$$
$$c^{(6)}_2=\frac{2}{6}\sum_{k=0}^{5}\cos\ab(\frac{2\pi (k+1/2)}{6})f\ab[2\cos\ab(\frac{\pi(k+1/2)}{6})+1]$$
$$c^{(6)}_3=\frac{2}{6}\sum_{k=0}^{5}\cos\ab(\frac{3\pi (k+1/2)}{6})f\ab[2\cos\ab(\frac{\pi(k+1/2)}{6})+1]$$
$$c^{(6)}_4=\frac{2}{6}\sum_{k=0}^{5}\cos\ab(\frac{4\pi (k+1/2)}{6})f\ab[2\cos\ab(\frac{\pi(k+1/2)}{6})+1]$$
$$c^{(6)}_5=\frac{2}{6}\sum_{k=0}^{5}\cos\ab(\frac{5\pi (k+1/2)}{6})f\ab[2\cos\ab(\frac{\pi(k+1/2)}{6})+1]$$
使用程序\file{3-1.py}計算得到:
過程略,結果如下:
$$S(x) = 9 x^5 - 5 x^4 - 3.67 x^3 + 14 x^2 + x - 12.0$$
\clearpage
% ==============================Problem 2==================================
\subsection{Pade近似(15pt)}
給出$f(x)=e^x$的$(2,2)$階Pade近似。



% ==============================Solution 2==================================    
\textbf{Solution:}先對$f(x)$用泰勒展開得到$c_0=1, c_1=1, c_2=\frac12, c_3=\frac16, c_4=\frac{1}{24}$,利用課件公式,有:
\begin{align*}
    b_1c_2 + b_2c_1 &= -c_3 \\
    b_1c_3 + b_2c_2 &= -c_4 \\
    b_0c_1 + b_1c_0 &= a_1 \\
    b_0c_2 + b_1c_1 + b_2c_0 &= a_2 \\
\end{align*}
解得$b_1=-\frac{1}{2}, b_2=\frac{1}{12}, a_1=\frac12, a_2=\frac{1}{12}$。
所以有
$$R(x)=\frac{1 + \frac{x}{2}+\frac{x^2}{12}}{1-\frac{x}{2}+\frac{x^2}{12}}$$
\clearpage
% ==============================Problem 3==================================
\subsection{數值積分(35pt)}
利用梯形法则、辛普森法则以及 Gauss-Legendre 方法,给出下面积分的数值结果:
$$\int_1^{100}\frac{e^{-x}}{x}\dd x$$
其中梯形法则、辛普森法的格点数分别取取为$10,100,1000$(格点包括左右端点)。Gauss Legendre 方法格点数为 $10,100$,Gauss-Legendre 节点和权重因子可以查阅文献或者调用已有的库函数,不用推算。

注:要求程序明确输出,并在答案文档中明确写出这几种情况下的计算结果,至少保留\textbf{五位}有效数字。 


% ==============================Solution 3==================================
\textbf{Solution:}
\clearpage

% ==============================Problem 4==================================
\subsection{样条函数在计算机绘图中的运用}



% ==============================Solution 4==================================
\textbf{Solution:}
\clearpage
% ==============================Problem 5==================================
\subsection{对称矩阵特征值问题}


% ==============================Solution 5==================================
\textbf{Solution:}
\clearpage


\end{document}